\documentclass{beamer}
\usepackage[utf8]{inputenc}
\usepackage[spanish]{babel}
\usetheme{Warsaw}
\usecolortheme{crane}
\useoutertheme{shadow}
\useinnertheme{rectangles}

\title[santosg572@gmail.com]{Sistema de Numeración}
%\subtitle{Dando nombres a los animales}
\author[L. González-Santos]{
L. González-Santos$^{1}$}
\institute[EDEN \& HELL]{
  $^{1}$
  Instituto de Neurobiología, UNAM\\
  Campus Juriquilla, Qro.
  \and
  \texttt{lgs@unam.mx}
}
\date{\today}

\begin{document}

\frame{\titlepage}

\begin{frame}
\frametitle{Sistema de numeración}

\textbf{Definición}

\hfill

Un \textbf{sistema de numeración} es un conjunto de símbolos y reglas de generación que permiten construir todos los números 
válidos. 

\hfill

Un sistema de numeración puede obtenerse como:

$\mathcal{N} =(\mathcal {S}, \mathcal {R})$

donde:

\begin{tiny}
\begin{itemize}
\item $\mathcal{N}$ es el sistema de numeración considerado (p.ej. decimal, binario, hexadecimal, etc.).
\item $\mathcal{S}$ es el conjunto de símbolos permitidos en el sistema. En el caso del sistema decimal son {0,1,2...9}; en el 
binario son {0,1}; en el octal son {0,1,...7}; en el hexadecimal son {0,1,...9,A,B,C,D,E,F}.
\item $\mathcal{R}$ son las reglas que nos indican qué números y qué operaciones son válidos en el sistema, y cuáles no. En un 
sistema de numeración posicional las reglas son bastante simples, mientras que la numeración romana requiere reglas algo más 
elaboradas.
\end{itemize}
\end{tiny}
\end{frame}  

\begin{frame}
\frametitle{Sistemas de numeración posicionales}

https://es.wikipedia.org/wiki/Sistema_de_numeración

El número de símbolos permitidos en un sistema de numeración posicional se conoce como base del sistema de numeración. Si un 
sistema de numeración posicional tiene base b significa que disponemos de b símbolos diferentes para escribir los números, y 
que b unidades forman una unidad de orden superior.


\hfill

\textbf{Ejemplo en el sistema de numeración decimal}

\hfill

Si contamos desde 0, incrementando una unidad cada vez, al llegar a 9 unidades, hemos agotado los símbolos disponibles, y si 
queremos seguir contando no disponemos de un nuevo símbolo para representar la cantidad que hemos contado. Por tanto añadimos 
una nueva columna a la izquierda del número, reutilizamos los símbolos de que disponemos, decimos que tenemos una unidad de 
primer orden (decena), ponemos a cero las unidades, y seguimos contando.


\end{frame}

\begin{frame}
\frametitle{Teorema fundamental de la numeración}

Este teorema establece la forma general de construir números en un sistema de numeración posicional. Primero estableceremos 
unas definiciones básicas:

$N$, número válido en el sistema de numeración.

$b$, base del sistema de numeración. Número de símbolos permitidos en el sistema.

$d_{i}$, un símbolo cualquiera de los permitidos en el sistema de numeración.

$n$,número de dígitos de la parte entera.

$,$, coma fraccionaria. Símbolo utilizado para separar la parte entera de un número de su parte fraccionaria.

$k$, número de dígitos de la parte decimal.

\end{frame}

\begin{frame}
\frametitle{Teorema fundamental de la numeración}

La fórmula general para construir un número N, con un número finito de decimales, en un sistema de numeración posicional de 
base b es la siguiente:



\end{frame}

\begin{frame}
\frametitle{Operadores Aritmético, Comparación}

\end{frame}

\begin{frame}
\frametitle{Operadores Aritmético, Comparación}

\end{frame}

\begin{frame}
\frametitle{Operadores Aritmético, Comparación}

\end{frame}

\begin{frame}
\frametitle{Operadores Aritmético, Comparación}

\end{frame}

\begin{frame}
\frametitle{Operadores Aritmético, Comparación}

\end{frame}

\begin{frame}
\frametitle{Operadores Aritmético, Comparación}

\end{frame}

\begin{frame}
\frametitle{Operadores Aritmético, Comparación}

\end{frame}

\begin{frame}
\frametitle{Operadores Aritmético, Comparación}

\end{frame}

\begin{frame}
\frametitle{Operadores Aritmético, Comparación}

\end{frame}

\begin{frame}
\frametitle{Operadores Aritmético, Comparación}

\end{frame}

\begin{frame}
\frametitle{Operadores Aritmético, Comparación}

\end{frame}

\begin{frame}
\frametitle{Operadores Aritmético, Comparación}

\end{frame}

\begin{frame}
\frametitle{Operadores Aritmético, Comparación}

\end{frame}




\end{document}


